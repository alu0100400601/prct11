\documentclass{beamer}
\usepackage[spanish]{babel}
\usepackage[utf8]{inputenc}
\usepackage{graphicx}

\newcommand{\PI}{{$\pi$}}

\title[Presentación con Beamer]{Presentación sobre \PI usando BEAMER}
\author[Dácil]{Dácil Batista}
\institute{ULL}
\date[24-04-2014]{24 de abril de 2014}

\usetheme{Madrid}

\definecolor{MiVioleta}{RGB}{122,59,122}
\definecolor{MiAzul}{RGB}{0,88,147}
\definecolor{MiGris}{RGB}{56,61,66}
\setbeamercolor{palette primary}{use=structure,fg=white,bg=MiVioleta}
\setbeamercolor{palette secundary}{use=structure,fg=white,bg=MiAzul}
\setbeamercolor{palette tertiary}{use=structure,fg=white,bg=MiGris}

\begin{document}

\begin{frame}
\titlepage
\end{frame}

\begin{frame}
\frametitle{Indice}
\tableofcontents[pausesections]
\end{frame}

\section{Introducción}

\begin{frame}
\frametitle{Introducción}

El número pi, representado por la letra griega \PI, equivale a la constante que relaciona el perímetro o longitud de una circunferencia con su diámetro. Se trata de un valor con un infinito número de decimales, cuya secuencia comienza de la siguiente manera:

\centerline{3,1415926535897932384626433832795028841\dots}

Redondeado en 3,1416, pi es un número irracional, es decir, no puede representarse de forma fraccional. Frecuentemente utilizado en las matemáticas y en la física, además de en otras disciplinas como la geometría y la trigonometría.


\end{frame}

\section{Inicios del número \PI}
\begin{frame}
\frametitle{Inicios del número \PI}

Ya en la antigüedad, se insinuó que todos los círculos conservaban una estrecha dependencia entre el contorno y su radio pero tan sólo desde el siglo XVII la correlación se convirtió en un dígito y fue identificado con el nombre ''Pi''.

Esta notación fue usada por primera vez en 1706 por el matemático galés William Jones y popularizada por el matemático Leonard Euler en su obra ''Introducción al cálculo infinitesimal'' de 1748. Fue conocida anteriormente como constante de Ludoph (en honor al matemático Ludolph van Ceulen) o como constante de Arquímedes.

El valor computado de esta constante ha sido conocido con diferentes precisiones en el curso de la historia, de esta forma en una de las referencias documentadas más antiguas como la Biblia aparece de forma indirecta asociada con el número natural 3, mientras en Mesopotamia los matemáticos la empleaban como 3 y una fracción añadida de $\frac{1}{8}$.

\end {frame}

\begin{frame}
\frametitle{Inicios del número \PI}

El número \PI  es una de las constantes matemáticas que más aparece en las ecuaciones de la física, junto con el número e, y es tal vez por ello la constante que más pasiones desata entre los matemáticos profesionales y aficionados.

Euclides precisa en sus Elementos los pasos al límite necesarios e investiga un sistema consistente en doblar, al igual que Antiphon, el número de lados de los polígonos regulares y en demostrar la convergencia del procedimiento.

Arquímedes reúne y amplía estos resultados. Prueba que el área de un círculo es la mitad del producto de su radio por la circunferencia y que la relación del perímetro al diámetro está comprendida entre 3,14084 y 3,14285.

\end{frame}

\begin{frame}
\frametitle{Inicios del número \PI}

En el siglo II d. de C., Ptolomeo utiliza polígonos de hasta 720 lados y una circunferencia de 60 unidades de radio para  aproximarse un poco más, y da el valor:

\centerline{$3 + \frac{8}{60} + \frac {30}{3600}= \frac{377}{120} = 3'14166\dots$}

Conforme se han desarrollado las matemáticas, en sus diversas ramas, álgebra, cálculo,\dots, se han ido construyendo distintos artificios que permiten afinar cada vez más su valor.

Uno de los casos más curiosos de la historia fue el del matemático inglés William Shanks, el cual después de un trabajo de unos casi veinte años, obtuvo 707 decimales en 1853. Desgraciadamente, Shanks incurrió en un error en el decimal que ocupa el lugar 528, y a partir de ese están todos mal.\cite{link1}

\end {frame}

\begin{frame}
\frametitle{Inicios del número \PI}
Desde esa fecha hacia delante, se han conseguido los siguientes resultados en la búsqueda de un valor para \PI:

\begin{itemize}
\item Ferguson, en 1947, obtuvo un valor con 808 decimales.
\item Usando el ordenador Pegasus, en 1597, se logró una cifra con 7840 decimales.
\item Más tarde, en 1961, usando un ordenador IBM 7090, se logró llegar a 100000 decimales.
\item Luego, en 1967, con un CDC 6600, se llegó a 500000 decimales. 
\item En 1987, con un Cray-2, se obtuvo una cifra con 100000000 decimales para \PI
\item Y finalmente, en 1995, en la Universidad de Tokio, se llegó a un valor de \PI de 3,14\dots y se le agregan 4294960000 de decimales.
\end{itemize}

\end{frame}

\begin{frame}
\frametitle{Valores del número \PI}
Aquí tenemos una tabla que resume los distintos valores de \PI, a lo largo de los años:

\begin{tabular}{lcc}
Matemático o lugar & Año & Valor\\
\hline
La Biblia & - & 3\\
Papiro de Ahmes (Egipto) & 1650 a.C. & 3,16 \\
Tablilla de Susa (Babilonia) & 1600 a.C. & 3,125 \\
Bandhayana (India) & 500 a.C. & 3,09\\
Arquímedes de Siracusa & (287-212 a.C.) & entre $\frac{223}{71} y \frac{220}{70}$\\
Liu Hui (China) & 260 & 3,1416\\
Tsu Chung Chih & 480 & entre 3,145926 y 3,1415927\\
Al-Kashi (Persia) & 1429 & 3,1415926535897932\\
Franciscus Vieta (Francia) & (1590-1603) & 3,1415926536
\end{tabular}

\end{frame}





\section{Aplicaciones del número \PI}
\begin{frame}
\frametitle{Aplicaciones del número \PI}

De todos los números utilizados en ciencias como las matemáticas o en ingenierías al igual que en la vida cotidiana, pocos despiertan tanto interés como el número \PI.

\end{frame}

\subsection{Geometría y Trigonometría}
\begin{frame}
\frametitle{Geometría y Trigonometría}
Para cualquier círculo de radio r y diámetro d = 2r, la longitud de la circunferencia es \PI$d$ y el área del círculo es \PI$r^2$. Además, \PI aparece en fórmulas para áreas y volúmenes de muchas otras figuras geométricas relacionadas con a circunferencia, como elipses, esferas, conos, y toroides\footnote{El toroide es la superficie de revolución generada por una curva plana cerrada que gira alrededor de una recta exterior coplanaria con la que no se interseca.}. El número \PI, aparece en integrales definidas que describen la circunferencia, área o volumen de figuras generadas por circunferencias y círculos.

De la definición de las funciones trigonométricas desde el círculo unitario se llega a que el seno y el coseno tienen período 2\PI. Además, el ángulo 180º es igual a \PI radianes. En otras palabras, un grado equivale a \PI  entre 180 radianes.

\end{frame}

\subsection{Física}
\begin{frame}
\frametitle{Física}
Aunque no es una constante física, \PI aparece rutinariamente en ecuaciones que describen los principios fundamentales del Universo. Debido en gran parte a su relación con la naturaleza del círculo y, correspondientemente, con el sistema de coordenadas esféricas. Usando unidades como las unidades de Planck se puede eliminar a veces a \PI de las fórmulas.

\end{frame}


\subsection{Probabilidad y estadística}
\begin{frame}
\frametitle{Probabilidad y estadística}
En probabilidad y estadística, hay muchas distribuciones cuyas fórmulas contienen a \PI, incluyendo:
\begin{itemize}
\item La función densidad de probabilidad para la distribución de Cauchy.

\item El problema de la aguja de Buffon es llamado en ocasiones como una aproximación empírica de \PI. Se trata de lanzar una aguja de longitud l repetidamente sobre una superficie en la que se han trazado rectas paralelas distanciadas entre sí, en t unidades, de manera uniforme (con t > l de forma que la aguja no pueda tocar dos rectas). Si la aguja se lanza n veces y x de esas cae cruzando una línea, entonces se puede aproximar \PI usando el Método de Montecarlo, el cual nos diría que \PI es aproximadamente $\frac{2nl}{xt}$.\cite{link2}
\end{itemize}

\end{frame}

\begin{frame}
\frametitle{Bibliografía}
\begin{thebibliography}
\beamertemplatebookbibitems
\bibitem[Guia Docente,2013]{guia}
Guia docente (Año 2013)
{\small $https://www.ull.es$}
%\bibitem

\end{thebibliography}
\end{frame}
\end{document}
